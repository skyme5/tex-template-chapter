\documentclass[]{scrbook}

\usepackage{tikz}
\usepackage{tikzpagenodes}
\usepackage{xcolor}
\usepackage{microtype}
\definecolor{myblue}{RGB}{0,82,155}

\begin{document}
	
	\begin{tikzpicture}[remember picture,overlay]
	\fill[gray!20]
	(current page.north west) rectangle ([yshift=-7cm]current page.north east);
	\node[inner sep=0pt,anchor=west] (label) at 
	([yshift=-4cm]current page text area.west|-current page.north)
	{\rotatebox[origin=c]{90}{%
			\normalfont\color{black}\Large%
			\textls[180]{\textsc{Chapter}}%
		}\hspace{10pt}%
		{\setlength\fboxsep{0pt}%
			\colorbox{myblue}{\parbox[c][3cm][c]{2.5cm}{%
					\centering\color{white}\fontsize{80}{90}\selectfont 2323}%
		}}%
	};
	\node[right of=label,fill=blue!25,text width=.5\paperwidth] at ([yshift=-6.5cm, xshift=.5\paperwidth]current page text area.west|-current page.north){\Huge This is a demonstration text for showing how line breaking works.};;
	\node[inner sep=0pt,anchor=north west] at ([yshift=-7cm]current page.north west)
	{\color{gray!20}\rule{\paperwidth}{1cm}};
	\end{tikzpicture}
	
\end{document}
